
\begin{preface}
Q-Leap is a large effort supported by the Japanese government, including major projects in quantum hardware, software, AI, and education or human resources development.  This book is a compilation of the contents of lessons on quantum communication from the Q-Leap Education project known as Quantum Academy of Science and Technology~\footnote{\url{https://qacademy.jp/en/}}. Quoting the website, "The Quantum Academy of Science and Technology operates as a common core program of the Q-LEAP Program of the Ministry of Education, Culture, Sports, Science and Technology based on 'Development of the Quantum Academy of Science and Technology Standard Program'."

The Quantum Academy aims to produce over thirty modules of similar size over a six-year period, with the work being done by the National Institute of Informatics (NII), University of Tokyo, Nagoya University, Kyushu University, and Keio University.  These thirty-plus modules will collectively comprise about one-quarter of an undergraduate degree, enough to build a major in quantum engineering for universities that choose to do so. Within this overall project, Keio's responsibility is quantum communications.  This book represents the first module delivered by AQUA, the research group of Prof. Rodney Van Meter at the Keio University Shonan Fujisawa Campus (SFC).  This module will be followed by "From Classical to Quantum Light" and "Quantum Internet" as the next two modules.  Each module will be available first as an online course, then later as a CC-BY-SA book.

This book itself, the videos and slides, and accompanying software demos are all licensed Creative Commons. You are not only \emph{allowed} to reuse these materials, you are \emph{encouraged} to do so, provided you credit the original authors, the AQUA team, and our institution, Keio University.

If you want to recompile the \LaTeX, the source is available on Github. You are free to reorder, remove or add chapters to suit your needs for your particular purposes.  More suggestions on how to use the material are in the next section.

\author{Michal and Rodney}
\date{sometime in 2023}
\end{preface}

